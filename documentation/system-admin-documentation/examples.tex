\section{Customization}
\label{sec:examples}
Possible changes to {\germinate} may range from basic formatting changes to the modification of existing content. The following sections contain detailed examples for the most common scenarios of extension. Each example consists of the necessary code and accompanying explanations.

\subsection{Internationalizing your pages}
\label{sec:example_i18n}
As already mentioned in Section \ref{sec:features_i18n}, {\germinate} supports internationalization for as many languages as you want.
Internationalizing your custom code is really easy:
\begin{enumerate}
    \item Define a method that returns your localized text in \path{jhi.germinate.client.i18n.Text}.
    \item Add the localized values for this new variable to the \texttt{.properties} files located at\\\instanceStuff\texttt{/<project.name>/modules/core/}. The file naming convention is: \texttt{Text\textunderscore\allowbreak <Language>\allowbreak \textunderscore <Region>.properties}.
\end{enumerate}
\noindent
A list of supported locales is available at \cite{JavaLocales}.

As an example, we will now add a new localized String to the application. To get the localized text, we used the method \texttt{Text.LANG.menuData()} which returned "data" for the English version of the page. Now let's assume we want to add a menu item that has the text "my fancy page". The first thing we have to do is create a new method in the interface \path{jhi.germinate.client.i18n.Text}:
\begin{lstlisting}[style=Java]
String menuMyFanyPage();
\end{lstlisting}
\noindent
As we can see, this method will return a \texttt{String} which we can use in our menu. You can regard this method as a template for the localization.\\
\\
The second step is to provide a translation of this text for each of the supported languages:
\begin{lstlisting}[style=Properties]
menuMyFanyPage=my fancy page
\end{lstlisting}
Add this to the files mentioned in the second bullet point above.\\
\\
GWT will substitute the template calls with the appropriate localized text during compile time.

\subsubsection{Parametrization and Plural Forms}
Instead of just returning static text from the properties files, {\germinate} is able to substitute place holders with parameters allowing for a dynamic usage of localization. One specific case of parameter usage is the case of plural forms. The text you want to display might differ based on the number of menus it should represent. As an example, consider the sentence: "I can see 3 trees". The sentence changes for just one tree: "I can see a tree". The method returning this text could look like this:

\begin{lstlisting}[style=Java]
String iSeeTrees(@PluralCount(DefaultRule_en.class) int number);
\end{lstlisting}
\noindent
The annotation \texttt{@PluralCount} tells the method that the parameter called \texttt{number} is a countable variable.
The associated properties entries could look like this:
\begin{lstlisting}[style=Properties]
iSeeTrees=I can see {0} trees.
iSeeTrees[one]=I can see a tree.
iSeeTrees[many]=I can see many trees.
\end{lstlisting}
As you can see, the plural form is handled by the first line, while the singular form is handled by the second line. A list of available plural forms can be seen in Table \ref{tab:examples_plural_form}. For more information, please refer to \cite{GWTPluralForms}.

\begin{table}
    \centering
    \caption{Available plural form rules in GWT}
    \label{tab:examples_plural_form}
    \begin{tabular}{rll}
	    \toprule
	    \textbf{Category} & \textbf{Rules} & \textbf{Examples}\\
	    \midrule
	    \textbf{zero} & $x$ is $0$ & 0\\
	    \textbf{one} & $x$ is $1$ & 1\\
	    \textbf{two} & $x$ is $2$ & 2\\
	    \textbf{few} & $x \bmod 100 \in \{3,...,10\}$ & 3-10, 103-110, \dots\\
	    \textbf{many} & $x \bmod 100 \in \{11,...,99\}$ & 11-99, 111-199, \dots\\
	    \textbf{other} & Everything else & 100-102, 200-202, 11.68, \dots\\
	    \bottomrule
    \end{tabular}
\end{table}

Another case of parametrization is simple substitution. As an example we will create a method returning a welcome message for the user and the day of the week.

\begin{lstlisting}[style=Java]
String welcomeMessage(String username, String day);
\end{lstlisting}
\noindent
The method is defined just the way a normal method taking two parameters would be defined. The localized entry in the properties file could look like this:

\begin{lstlisting}[style=Properties]
welcomeMessage=Hello {0}. Welcome to {\germinate}. Today is {1}.
\end{lstlisting}
\noindent
Calling \texttt{welcomeMessage("Joe Bloggs", "Wednesday");} will result in "Hello Joe Bloggs. Welcome to {\germinate}. Today is Wednesday".

\subsubsection{Notes}
\begin{itemize}
    \item It is possible to include HTML tags in the localized strings. Consequently, you can change the font style (size, bold, italic,...), add hyperlinks, include images, etc.
    \item Use two single quotes instead of just one (\eg "It''s done" instead of "It's done").
    \item The encoding of the properties files has to be UTF-8.
\end{itemize}

\subsection{Adding a new language}
\label{sec:i18n}
We have seen how to add content to existing languages in the previous section. Now we will show how to add a completely new language to the application. To ensure neutrality, we selected Switzerland and will use the language of Swiss German. The locale id of Swiss German is \texttt{de\textunderscore CH}. The first part represents the language (de = Deutsch which is German for "German") and the second part represents the country (CH = Confoederatio Helvetica which is Latin for "Swiss Confederation").

Navigate to your configuration folder located at \instanceStuff\texttt{/<project.name>} and create a new file called \texttt{Text\textunderscore de\textunderscore CH.properties}. Now copy the whole content of one of the fully translated other languages into this file and substitute all of the text with its appropriate translation.

Next, you'll need to make a change to this file: \path{war/css/language-selector-css.jsp}. Search for the line containing the 2-digit country code of the country associated with the new language. In our example, that would be \texttt{CH} and the line looks something like this:

\begin{lstlisting}[style=CSS]
span.country.ch { background-position: 0px -1440px; }
\end{lstlisting}
\noindent
Add a new line with your new locale before this line. Mind the comma:

\begin{lstlisting}[style=CSS]
.language-selector span.de_CH,
span.country.ch { background-position: 0px -1440px; }
\end{lstlisting}
\noindent
Finally, open the file \instanceStuff\texttt{/<project.name>/Germinate.gwt.xml} and add this line:

\begin{lstlisting}[style=Xml]
<extend-property name="locale" values="de_CH" />
\end{lstlisting}
\noindent
After compiling and deploying the project, you should be able to select the new localization from the language selector at the top of the page.

\subsubsection{Notes}
\begin{itemize}
    \item The default locale is determined by the browser settings. A fall-back solution can be specified by this entry in the \texttt{Germinate.gwt.xml} file:
            \begin{lstlisting}[style=Xml]
<set-property-fallback name="locale" value="en_GB" />
            \end{lstlisting}
    \item Not all localization files have to be complete. If translations are missing, they will be substituted by their respective value from the default locale.
\end{itemize}

\subsection{Internationalized files and images}
\label{sec:localized-files}
In some cases it may be necessary to supply a download file or image in more than just one language. We decided to adopt a concept called \textit{resource qualifiers}. This concept is also used in the popular Android platform \cite{ResourceQualifiers}.

The main idea of this concept is to provide resources in alternate forms which will be used under certain circumstances. On Android one example of a resource qualifier are language and region. Valid examples are \textit{de} for German resources or \textit{en-rGB} for British English.

The advantage of this concept is that you only have to copy your files/images to the appropriate folder and the system will choose the correct file based on the current configuration. If a file does not exist in the folder for the current configuration, the system will fall back to the default file.

We adopted this idea for {\germinate}. We allow internationalization of files in the directories "download", "data", and "res" (see Section \ref{sec:structure} for details). To provide internationalized versions of your files, you need to create a new directory of the type you want to extend (e.g. "download") and then append the locale separated by a dash. In the previous sections we used the local \texttt{de\textunderscore CH} as an example. The internationalized download directory for this locale would have to be named \texttt{download-de\textunderscore CH}. The structure within this directory has to be identical to the base-directory ("download"). Please note, that this concept only works if the files in the different directories have the same file name. Otherwise, {\germinate} won't be able to find them.

The result for the user is a seamless internationalization of all the content within {\germinate}. When changing the locale on the web interface, the server backend will serve the internationalized files for this locale, if available, and fall back to the default locale otherwise. 

\subsubsection{Passing parameters via the URL}
When {\germinate} is first loaded by the browser, it will parse the given URL parameters and save the respective parameters in the \texttt{ParameterStores}. As a result, you can create URLs that take the user to a specific page for a specific combination of parameters. As an example, the URL
\begin{center}
\texttt{http://<your\textunderscore server>:8080/<project.name>/?accessionId=1\#passport}
\end{center}
will take you to the passport page of {\germinate} showing the accession with id 1. Alternatively, the URL
\begin{center}
\texttt{http://<your\textunderscore server>:8080/<project.name>/?searchString=Baz\#search}
\end{center}
will show the search results for the given search string. Multiple parameters can be specified by combining them with an ampersand (\&).\\
\\
For security reasons, we only save valid parameters, \ie parameters that are part of the enum \texttt{Parameter}. Moreover, it is very important that the parameters are located \textbf{before} the fragment identifier (\#).


\subsubsection{Custom Bootstrap Theme}
\todo{TODO}


